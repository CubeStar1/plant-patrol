\section{Introduction}
\label{sec:introduction}

\subsection{The Challenge of Pest Management in Agriculture}
Agriculture is a foundational sector for global food security and economic stability. However, crop yields are constantly threatened by insect pests, which lead to significant losses each year. Traditional pest detection methods rely on manual inspection by experts, a process that is both time-consuming and inaccessible to many farmers, especially in remote or resource-limited regions.

\subsection{AI-Powered Solution for Pest Detection}
Recent advances in artificial intelligence and computer vision offer a promising alternative. Automated pest detection using deep learning can provide rapid, accurate, and scalable solutions to identify pests in real time. Among these, the YOLO (You Only Look Once) family of object detection algorithms stands out for its balance of speed and accuracy, making it ideal for real-time agricultural applications.

This project aims to bridge the gap between AI research and practical agricultural needs by developing a web-based application that leverages a custom-trained YOLOv11 model for insect pest detection. The application supports multiple input sources—including webcam feeds, uploaded images, and video files—making it versatile for field and laboratory use. By enabling early and precise pest identification, the tool empowers farmers to take timely action, reducing crop losses and promoting sustainable agricultural practices.

\subsection{Project Objectives}
The main objectives of this project are:
\begin{itemize}
    \item To design an accessible, user-friendly platform for real-time pest detection in agriculture.
    \item To utilize advanced deep learning techniques for robust and efficient pest identification.
    \item To support multiple input modalities for flexible deployment in diverse environments.
    \item To contribute to sustainable agriculture by enabling early intervention and reducing unnecessary pesticide use.
\end{itemize}