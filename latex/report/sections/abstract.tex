
Agricultural pest management is a critical factor in ensuring crop yields and food security. Traditional pest identification methods, which rely on manual inspection and expert knowledge, are time-consuming, labor-intensive, and often inaccessible to many farmers. Recent advancements in computer vision and deep learning have enabled the development of automated pest detection systems; however, many existing solutions lack real-time capabilities or are not easily accessible to end-users. This project addresses these gaps by proposing a real-time, web-based pest detection system utilizing a custom-trained YOLOv11 model, with the objective of providing an accurate, scalable, and user-friendly tool for rapid pest identification in agricultural settings.

The methodology involves the creation of a comprehensive web application that integrates a YOLOv11 deep learning model, trained on the "Insect Pest Detection in Agriculture using YOLO" dataset, which contains over 34,000 annotated images of 102 pest species. Data preprocessing steps such as image resizing, normalization, and augmentation were applied to improve model robustness. The model was trained using transfer learning and optimized hyperparameters to enhance detection accuracy. For deployment, the model was converted to ONNX format and integrated into a browser-based inference pipeline using ONNX Runtime Web, enabling real-time pest detection from webcam feeds, image uploads, and video files. The application interface, built with Next.js, React, and TypeScript, provides intuitive visualization and performance metrics, making it accessible on a wide range of devices.

Experimental results demonstrate that the proposed system achieves high detection accuracy and real-time performance suitable for field deployment. The trained YOLOv11 model attained a mean Average Precision (mAP@0.5) of 0.87, with precision and recall values ranging from 0.85 to 0.92 and 0.82 to 0.89, respectively. The optimized model processes input at 16--22 frames per second with an average inference time of 45--60 milliseconds per frame, while maintaining a compact size of 12.3 MB and memory usage below 200 MB. These results compare favorably to existing literature, indicating that the system is both efficient and reliable for practical agricultural applications.

In conclusion, the developed web-based pest detection platform offers a significant advancement in precision agriculture by providing farmers and agricultural professionals with a rapid, accurate, and accessible tool for pest identification. The system's modular design and real-time capabilities make it suitable for integration into broader crop management workflows. Future work will focus on expanding the pest species database, enhancing detection performance under challenging field conditions, and exploring integration with IoT devices for automated pest monitoring and smart agriculture solutions.
\end{abstract}
